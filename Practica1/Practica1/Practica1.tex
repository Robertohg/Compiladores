%package list
\documentclass{article}
\usepackage[top=3cm, bottom=3cm, outer=3cm, inner=3cm]{geometry}
\usepackage{graphicx}
\usepackage{url}
%\usepackage{cite}
\usepackage{hyperref}
\usepackage{array}
%\usepackage{multicol}
\newcolumntype{x}[1]{>{\centering\arraybackslash\hspace{0pt}}p{#1}}
\usepackage{natbib}
\usepackage{pdfpages}
\usepackage{multirow}
\usepackage{multirow}
\usepackage[normalem]{ulem}
\useunder{\uline}{\ul}{}


%%%%%%%%%%%%%%%%%%%%%%%%%%%%%%%%%%%%%%%%%%%%%%%%%%%%%%%%%%%%%%%%%%%%%%%%%%%%
%%%%%%%%%%%%%%%%%%%%%%%%%%%%%%%%%%%%%%%%%%%%%%%%%%%%%%%%%%%%%%%%%%%%%%%%%%%%
\newcommand{\csemail}{vmachacaa@ulasalle.edu.pe}
\newcommand{\csdocente}{MSc. Vicente Enrique Machaca Arceda}
\newcommand{\cscurso}{Compiladores}
\newcommand{\csuniversidad}{Universidad La Salle}
\newcommand{\csescuela}{Escuela Profesional de Ingeniería de Software}
\newcommand{\cspracnr}{01}
\newcommand{\cstema}{Introducción}
%%%%%%%%%%%%%%%%%%%%%%%%%%%%%%%%%%%%%%%%%%%%%%%%%%%%%%%%%%%%%%%%%%%%%%%%%%%%
%%%%%%%%%%%%%%%%%%%%%%%%%%%%%%%%%%%%%%%%%%%%%%%%%%%%%%%%%%%%%%%%%%%%%%%%%%%%


\usepackage[english,spanish]{babel}
\usepackage[utf8]{inputenc}
\AtBeginDocument{\selectlanguage{spanish}}
\renewcommand{\figurename}{Figura}
\renewcommand{\refname}{Referencias}
\renewcommand{\tablename}{Tabla} %esto no funciona cuando se usa babel
\AtBeginDocument{%
	\renewcommand\tablename{Tabla}
}

\usepackage{fancyhdr}
\pagestyle{fancy}
\fancyhf{}
\setlength{\headheight}{30pt}
\renewcommand{\headrulewidth}{1pt}
\renewcommand{\footrulewidth}{1pt}
\fancyhead[L]{\raisebox{-0.2\height}{\includegraphics[width=3cm]{img/logo_salle}}}
\fancyhead[C]{}
\fancyhead[R]{\fontsize{7}{7}\selectfont	\csuniversidad \\ \csescuela \\ \textbf{\cscurso} }
\fancyfoot[L]{MSc. Vicente Machaca}
\fancyfoot[C]{\cscurso}
\fancyfoot[R]{Página \thepage}

\usepackage{listings}
\usepackage{xcolor} % for setting colors

% set the default code style
\lstset{
    frame=tb, % draw a frame at the top and bottom of the code block
    tabsize=4, % tab space width
    showstringspaces=false, % don't mark spaces in strings
    numbers=left, % display line numbers on the left
    commentstyle=\color{green}, % comment color
    keywordstyle=\color{blue}, % keyword color
    stringstyle=\color{red} % string color
}






\begin{document}
	
	\vspace*{10px}
	
	\begin{center}	
		\fontsize{17}{17} \textbf{ Práctica \cspracnr}
	\end{center}
	%\centerline{\textbf{\underline{\Large Título: Informe de revisión del estado del arte}}}
	%\vspace*{0.5cm}
	

	\begin{table}[h]
		\begin{tabular}{|x{4.7cm}|x{4.8cm}|x{4.8cm}|}
			\hline 
			\textbf{DOCENTE} & \textbf{CARRERA}  & \textbf{CURSO}   \\
			\hline 
			\csdocente & \csescuela & \cscurso    \\
			\hline 
		\end{tabular}
	\end{table}	
	
	
	\begin{table}[h]
		\begin{tabular}{|x{4.7cm}|x{4.8cm}|x{4.8cm}|}
			\hline 
			\textbf{PRÁCTICA} & \textbf{TEMA}  & \textbf{DURACIÓN}   \\
			\hline 
			\cspracnr & \cstema & 3 horas   \\
			\hline 
		\end{tabular}
	\end{table}
	
	
	\section{Datos de los estudiantes}
	\begin{itemize}
		\item Grupo: 4
		\item Integrantes: 
		\begin{itemize}
			
			\item Gabriela Pacco Huamani
			\item Augusto Delgado Bravo
			\item Roberto Heredia Garland
		\end{itemize}		
	\end{itemize}
	
	
	

	
	\section{Ejercicios}\label{sec:ejercicios}
	\begin{enumerate}
		\item Redacta el siguiente código, genera el código ensamblador y explica en qué parte (del código
ensamblador) se definen las variables c y m. (2 puntos).

            
		
		Solución \\
		
        
\begin{lstlisting}[language={[x86masm]Assembler}, basicstyle=\small]    
; 1 Aqui se define la variable "m".
     .LFB0:
	.cfi_startproc
	endbr64
	pushq	%rbp
	.cfi_def_cfa_offset 16
	.cfi_offset 6, -16
	movq	%rsp, %rbp
	.cfi_def_cfa_register 6
	movl	$11148, -4(%rbp)
	movl	$0, %eax
	popq	%rbp
	.cfi_def_cfa 7, 8
	ret
	.cfi_endproc
	;Aqui se define  la  variable c:
	.LC0:
	.string	"abcdef"
	.text
	.globl	main
	.type	main, @function
 \end{lstlisting}	
	
		
		\item Redacta el siguiente código, genera el código ensamblador y explica en qué parte (del código
ensamblador) se define la división entre 8. (2 puntos).

Solución \\
\begin{lstlisting}[language={[x86masm]Assembler}, basicstyle=\small]	
	
	.LFE0:
	.size	main, .-main
	.ident	"GCC: (Ubuntu 9.3.0-10ubuntu2) 9.3.0"
	.section	.note.GNU-stack,"",@progbits
	.section	.note.gnu.property,"a"
	.align 8
	.long	 1f - 0f
	.long	 4f - 1f
	.long	 5
0:
	.string	 "GNU"
1:
	.align 8
	.long	 0xc0000002
	.long	 3f - 2f
2:
	.long	 0x3
3:
	.align 8
4:
 \end{lstlisting}
		\item Redacta el siguiente código, genera el código ensamblador y explica en qué parte (del código
ensamblador) se define la división entre 4. (2 puntos).

		Solución \\
		.................
		\item  Redacta el siguiente código, genera el código ensamblador y explica en qué parte (del código
ensamblador) se define la división entre 2. (2 puntos).
       
       Solución \\
		.................
		\item  Redacta el siguiente código, genera el código ensamblador y explica: (4 puntos):
		
		•	En qué parte del código ensamblador se define la función div4.
	\begin{lstlisting}[language={[x86masm]Assembler}, basicstyle=\small]
	Ltext0:
	.cfi_sections	.debug_frame
	.globl	__Z4div4i
	.def	__Z4div4i;	.scl	2;	.type	32;	.endef
	  \end{lstlisting}	
•	En qué parte del código ensamblador se invoca a la función div4.\newline
\begin{lstlisting}[language={[x86masm]Assembler}, basicstyle=\small]
	call	__Z4div4i
 \end{lstlisting}	
•	En qué parte del código ensamblador dentro de la función div4 se procesa la división.
\begin{lstlisting}[language={[x86masm]Assembler}, basicstyle=\small]
	movl	8(%ebp), %eax
	leal	3(%eax), %edx
	testl	%eax, %eax
	cmovs	%edx, %eax
	sarl	$2, %eax
\end{lstlisting}


    	\item  Redacta el siguiente código, genera el código ensamblador y explica: (4 puntos):
		
		•	En qué parte del código ensamblador se define la función div.
\begin{lstlisting}[language={[x86masm]Assembler}, basicstyle=\small]
	Ltext0:
	.cfi_sections	.debug_frame
	.globl	__Z3divii
	.def	__Z3divii;	.scl	2;	.type	32;	.endef
	 \end{lstlisting}
•	En qué parte del código ensamblador se invoca a la función div.
\begin{lstlisting}[language={[x86masm]Assembler}, basicstyle=\small]
call	__Z3divii
\end{lstlisting}
•	En qué parte del código ensamblador dentro de la función div se procesa la división.
\begin{lstlisting}[language={[x86masm]Assembler}, basicstyle=\small]
idivl	12(%ebp)
\end{lstlisting}


 	 \item De las preguntas anteriores, se ha generado código, por cada función, ambas dividen entre 4,
pero difieren un poco en su implementación. Investigue a qué se debe dicha diferencia y comente cuáles podrían ser las consecuencias. (4 puntos)

\begin{itemize}
    \item En la funcion $div$ se utiliza $idivl$ la cual es una instruccion para dividir signed numbers, por eso se utiliza cltd (lo cual convierte longs a dobuble longs
\begin{lstlisting}[language={[x86masm]Assembler}, basicstyle=\small]
cltd
idivl	12(%ebp)
\end{lstlisting}

\item En cambio en $div4$ se utiliza una operacion bitwise al dividir (Shift Arithmetic Right)


\begin{lstlisting}[language={[x86masm]Assembler}, basicstyle=\small]
	sarl	$2, %eax
\end{lstlisting}
\end{itemize}




	\end{enumerate}


	
	%\clearpage
	%\bibliographystyle{apalike}
	%\bibliographystyle{IEEEtranN}
	%\bibliography{bibliography}
		
	
\end{document}